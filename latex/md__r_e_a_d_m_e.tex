Muse\+Score is an open source and free music notation software. For support, contribution, bug reports, visit \href{https://musescore.org}{\tt Muse\+Score.\+org}. Fork and make pull requests!

\subsection*{Features}


\begin{DoxyItemize}
\item W\+Y\+S\+I\+W\+YG design, notes are entered on a \char`\"{}virtual notepaper\char`\"{}
\item True\+Type \hyperlink{structfont}{font(s)} for printing \& display allows for high quality scaling to all sizes
\item easy \& fast note entry
\item many editing functions
\item Music\+X\+ML import/export
\item Midi (S\+MF) import/export
\item Muse\+Data import
\item Midi input for note entry
\item integrated sequencer and software synthesizer to play the score
\item print or create pdf files
\end{DoxyItemize}

\subsection*{More info}


\begin{DoxyItemize}
\item \href{https://musescore.org}{\tt Muse\+Score Homepage}
\item \href{https://musescore.org/en/developers-handbook/git-workflow}{\tt Muse\+Score Git workflow instructions}.
\item \href{https://musescore.org/en/developers-handbook/compilation}{\tt How to compile Muse\+Score?}
\item Build status\+: \href{https://travis-ci.org/musescore/MuseScore}{\tt }
\end{DoxyItemize}

\subsection*{License}

Muse\+Score is licensed under G\+PL version 2.\+0. See L\+I\+C\+E\+N\+S\+E.\+G\+PL in the same directory.

\subsection*{Packages}


\begin{DoxyItemize}
\item {\bfseries aeolus} Clone of \href{http://kokkinizita.linuxaudio.org/linuxaudio/aeolus/}{\tt Aeolus} Disabled by default in the stable releases. See \href{http://dev-list.musescore.org/Aeolus-Organ-Synth-td7578364.html}{\tt http\+://dev-\/list.\+musescore.\+org/\+Aeolus-\/\+Organ-\/\+Synth-\/td7578364.\+html} Kept as an example of how to integrate with a complex synthesizer.
\item {\bfseries assets} Graphical assets, use them if you need a Muse\+Score icon. For logo, color etc... see \href{https://musescore.org/en/about/logos-and-graphics}{\tt https\+://musescore.\+org/en/about/logos-\/and-\/graphics}
\item {\bfseries awl} Audio Widget Library, from the MusE project
\item {\bfseries build} Utility files for build
\item {\bfseries bww2mxml} Command line tool to convert B\+WW files to Music\+X\+ML. B\+WW parser is used by Muse\+Score to import B\+WW files.
\item {\bfseries demos} A few Muse\+Score files to demonstrate what can be done
\item {\bfseries fluid} Clone of \href{https://sourceforge.net/projects/fluidsynth/}{\tt Fluid\+Synth}, ported to C++ and customized
\item {\bfseries fonts} Contains fontforge source (sfd) + ttf/otf fonts. Muse\+Score includes the \char`\"{}\+Emmentaler\char`\"{} font from the Lilypond project.
\item {\bfseries libmscore} Data model of Muse\+Score
\item {\bfseries mscore} Main code for the Muse\+Score UI
\item {\bfseries mstyle} Clone of K\+D\+E4 style Oxygen
\item {\bfseries msynth} Abstract interface to Fluid + \hyperlink{class_aeolus}{Aeolus}
\item {\bfseries mtest} Unit testing using Q\+Test
\item {\bfseries omr} Optical music recognition
\item {\bfseries share} Files moved to /usr/share/... on install
\item {\bfseries test} Old tests. Should move to mtest
\item {\bfseries vtest} Visual tests. Compare reference images with current implementation
\item {\bfseries thirdparty} Contains projects which are included for convenience, usually to integrate them into the build system to make them available for all supported platforms.
\begin{DoxyItemize}
\item {\bfseries thirdparty/rtf2html} Used for capella import
\item {\bfseries thirdparty/diff} Not used currently. \href{https://code.google.com/p/google-diff-match-patch/}{\tt Diff, Match and Patch Library}
\item {\bfseries thirdparty/ofqf} O\+SC server interface. Based on \href{http://www.arnoldarts.de/projects/ofqf/}{\tt O\+SC for Qt4}
\item {\bfseries thirdparty/singleapp} Clone from \href{https://github.com/qtproject/qt-solutions/tree/master/qtsingleapplication}{\tt Qt Single Application}
\item {\bfseries thirdparty/portmidi} Clone from \href{https://sourceforge.net/projects/portmedia/}{\tt Port\+Midi}
\item {\bfseries thirdparty/beatroot} It\textquotesingle{}s a core part of Beat\+Root Vamp Plugin by Simon Dixon and Chris Cannam, used in M\+I\+DI import for beat detection. (\href{https://code.soundsoftware.ac.uk/projects/beatroot-vamp/repository}{\tt https\+://code.\+soundsoftware.\+ac.\+uk/projects/beatroot-\/vamp/repository})
\end{DoxyItemize}
\end{DoxyItemize}

\subsection*{Building}

{\bfseries Read the developer handbook for a \href{https://musescore.org/en/developers-handbook/compilation}{\tt complete build walkthrough} and a list of dependencies.}

\subsubsection*{Getting sources}

If using git to download repo of entire code history, type\+: \begin{DoxyVerb}git clone https://github.com/musescore/MuseScore.git
cd MuseScore
\end{DoxyVerb}


Else can just download the latest source release tarball from \href{https://github.com/musescore/MuseScore/releases,}{\tt https\+://github.\+com/musescore/\+Muse\+Score/releases,} and then from your download directory type\+: \begin{DoxyVerb}tar xzf MuseScore-x.x.x.tar.gz
cd MuseScore-x.x.x
\end{DoxyVerb}


\subsubsection*{Release Build}

To compile Muse\+Score, type\+: \begin{DoxyVerb}make release
\end{DoxyVerb}


If something goes wrong, then remove the whole build subdirectory with {\ttfamily make clean} and start new with {\ttfamily make release}.

\subsubsection*{Running}

To start Muse\+Score, type\+: \begin{DoxyVerb}./build.release/mscore/mscore
\end{DoxyVerb}


The Start Center window will appear on every invocation, until you disable that setting via the \char`\"{}\+Preferences\char`\"{} dialog.

\subsubsection*{Installing}

To install to default prefix using root user, type\+: \begin{DoxyVerb}sudo make install
\end{DoxyVerb}


\subsubsection*{Debug Build}

A debug version can be built by doing {\ttfamily make debug} instead of {\ttfamily make release}.

To run the debug version, type\+: \begin{DoxyVerb}./build.debug/mscore/mscore
\end{DoxyVerb}


\subsubsection*{Testing}

See /mtest/\+R\+E\+A\+D\+ME.md \char`\"{}mtest/\+R\+E\+A\+D\+M\+E.\+md\char`\"{} or \href{https://musescore.org/en/developers-handbook/testing}{\tt https\+://musescore.\+org/en/developers-\/handbook/testing} for instructions on how to run the test suite. 